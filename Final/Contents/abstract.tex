\chapter*{Abstract}
\addcontentsline{toc}{chapter}{Abstract}
\thispagestyle{plain}
\pagenumbering{roman}
\setcounter{page}{4}
%\begin{sloppypar}
\sloppy
This project focuses on the detection of steganographically modified images using an ensemble classifier model. Steganography is a technique used to hide information within another carrier file, and it poses a serious threat to digital privacy and security. The proposed model utilizes a bagging method with FLD as base learner to train an ensemble classifier. High dimensional features from the CC-C300 model are used to train the classifier, which offers robustness, model diversity, and faster development compared to other machine learning methods. The model is expected to effectively detect steganographic methods such as nsF5, J-UNIWARD, and UERD, which hide messages in JPEG images. The project is technically feasible, economically feasible, and operationally feasible. The methodology involves data collection from the IStego100K dataset, feature extraction using JPEG coefficient analysis, and training of the ensemble classifier. The expected outcomes include the successful detection of steganographically modified images and the ability to classify them as cover or stego images. The project aims to provide a reliable and efficient solution to the growing threat of steganography in digital communication. \\ \\
\normalsize{\textbf{Keywords:} Steganography, Steganalysis, Ensemble Classifiers, CC-C300, \mbox{Machine Learning}, FLD , JPEG }
%\end{sloppypar}


