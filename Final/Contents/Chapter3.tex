\chapter{Requirements Analysis}
\section{Software Requirements}
Software requirements for the given project are as follows:
\begin{enumerate}[noitemsep]
    \item \textbf{Python:} 
    Python is a versatile programming language commonly used for \mbox{developing} software applications. It can be used for various tasks in the system, such as backend development, data processing, and machine learning integration.
    ementing interactive features on the system’s web interface, and facilitating communication with the backend. 
    \item \textbf{GitHub:}
    GitHub is a web-based platform for version control using Git. It provides a collaborative environment for software development projects, allowing developers to host and share their code, track changes, manage workflows, and collaborate with others. GitHub offers features such as code repositories, issue tracking, project management tools, code review, and continuous integration.
    \item \textbf{MATLAB:}
    MATLAB, short for ``Matrix Laboratory,'' is a high-level programming language and interactive environment primarily used for numerical computation, visualization, and programming. It provides a wide range of built-in functions and toolboxes for various applications, including mathematics, signal processing, image processing, control systems, and machine learning.
    \item \textbf{VS Code:}
    VS Code is a popular and widely used source code editor that offers a range of features and extensions to enhance the development experience. It supports multiple programming languages, including Python, JavaScript, and React, making it suitable for working with the different components of the system. 
    \item \textbf{LaTeX:}
    LaTeX is a typesetting system used for creating documents, particularly those that require complex mathematical equations or scientific notation. It is widely used in academic and technical fields for its ability to produce high-quality documents with consistent formatting. LaTeX uses markup language to format text, and it is highly customizable, allowing users to create templates and styles for their documents. It is also free and open-source, making it accessible to anyone who wants to use it.
    \end{enumerate}
\section{Hardware Requirements}
Hardware requirements for the given project are as follows:
\begin{enumerate}[noitemsep] %label=\Roman*.]
    \item Employed systems equipped with 16 GB RAM, 3.6 GHz processors, and NVIDIA GPUs for optimal performance.
    \item Utilized a dedicated server having 64 GB RAM to enhance computational capabilities.
   
 \end{enumerate}

 \section{Functional Requirements}
 The functional requirements for the prepared project are as follows:
 \begin{enumerate}[noitemsep]
    \item The Final model must be able to distinguish between cover and stego images.
    \item The system must be able to process images in real-time.
    \item The UI must be user-friendly and responsive.
 \end{enumerate}

 \section{Non-Functional Requirements}
 The non-functional requirements for the prepared project are as follows:
 \begin{enumerate}[noitemsep]
    \item \textbf{Reliability:} The system must be reliable and consistent in its performance.
    \item \textbf{Maintainability:} The system must be easy to maintain and update.
    \item \textbf{performance:} The system must be able to process images quickly and efficiently.
    \item \textbf{Accuracy:} The system must be able to accurately detect stego images.
    \item \textbf{Compatibility:} The system must be compatible with different operating systems and devices.
 \end{enumerate}