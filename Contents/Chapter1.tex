\chapter{Introduction}
\pagenumbering{arabic}
\section{Background}
Steganography is like hiding a secret message, like a picture or music. It's a way of keeping your message private by making it blend in, so others don't even realize there's a secret there. Steganalysis, the detection of hidden information within digital media, is crucial for maintaining the integrity and security of digital communication. Uncovering hidden information is vital for maintaining the security of digital communication channels, preventing covert communication that may pose risks.\vspace{0.5cm}\\
\textbf{Image Steganography}\\
Image Steganography is the process of hiding information which can be text, image or video inside a cover image. The secret information is hidden in a way that is not visible to the human eyes.
Different techniques of image steganography:
\begin{itemize}[noitemsep]
\item \textbf{nsF5}
\item \textbf{UERD(uniform embedding revisited distortion)}
\item \textbf{J-Uniward}\\
\end{itemize}
Steganography is an ever-evolving science of concealing information, continuously evolving to counteract detection methodologies. In response to this continuous evolution, the deployment of an automatically adaptive detection system becomes necessary. Embedding machine learning within steganalysis emerges as an optimal strategy to effectively counter the continuous evolution of covert communication methods.\\
\clearpage
\section{Problem Statement}
The continuous evolution of steganographic techniques poses a critical challenge to digital security. With the increasing sophistication of methods used to embed information covertly, traditional steganalysis approaches are challenged by the need for improved accuracy and adaptability. Since two-thirds of the internet is composed of images and images play a major role in digital communication, the use of advanced steganographic techniques to hide malicious data inside another carrier information poses an intricate threat to the security of a system. Creative approaches are required since secretly implanted malicious programs are difficult for traditional steganalysis to accurately identify. The complexity of compression and encryption methods adds an additional level of difficulty in detection. Existing steganalysis, which was created for traditional steganography, is not flexible enough to detect the complex algorithms for steganography. This proposal aims to fill these gaps by developing ensemble classifiers based steganalysis models that will enhance accuracy and sensitivity of steganalysis. Thus, to protect the integrity of digital communication, this proposal seeks to propose an steganalysis process created with the help of ML to detect and tackle the subtle changes caused by the concealing of malicious data within unsuspecting carrier files.
\clearpage
\section{Objectives}
The main objectives of this project is to:
\begin{itemize}
\item \textbf{Diverse Steganalysis:} To create steganalysis models capable of detecting concealed images across various  resolutions and compression methods.
\item \textbf{Diverse Algorithmic Approach:} To explore novel algorithms and methodologies tailored for analyzing intricate patterns in image-in-image steganography.
\end{itemize}

