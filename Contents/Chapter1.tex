\chapter{Introduction}
\pagenumbering{arabic}
\section{Background}
Steganography is like hiding a secret message, like a picture or music. It's a way of keeping your message private by making it blend in, so others don't even realize there's a secret there. It has many types such as : image steganography, audio steganography, video steganography.\\
\textbf{Image Steganography}\\
Image Steganography is the process of hiding information which can be text, image or video inside a cover image. The secret information is hidden in a way that is not visible to the human eyes.
Different techniques of image steganography:
\begin{itemize}
\item \textbf{Transform Domain Steganography}\\
These techniques involve transforming the image data into a different domain (e.g., frequency domain using Discrete Cosine Transform or wavelet domain using Discrete Wavelet Transform) and then hiding information in the transformed coefficients.\\
\item \textbf{Compressed Domain Steganography}\\
Hides information within the compressed data of an image file to reduce file size and detection difficulty.\\
\item \textbf{Least Significant Bit (LSB) Technique}\\
It involves replacing the least significant bits of the pixel values with the secret message bits.\\
\item \textbf {Pixel Value Differencing (PVD) Technique}\\
Identifies and modifies pixels with small value differences to encode information in both grayscale and color images.\\
\end{itemize}
\section{Problem Statement}
The development of image steganography poses a serious cybersecurity threat. Since images play a major role in digital communication, the use of advanced steganographic techniques to hide malicious data inside another carrier information pose an intricate threat to the security of a system. Creative approaches are required since, secretly implanted malicious programs are difficult for traditional steganalysis to accurately identify. The complexity of compression and encryption methods adds an additional level of difficulty in detection. Existing steganalysis which were created for traditional steganography, are not flexible enough to detect the ever evolving algorithms for steganography. Thus, in order to protect the integrity of digital communication. This proposal seeks to propose an ever evolving steganalysis process created with the help of ML to detect and tackle the subtle changes caused by concealing of malicious data within unsuspecting carrier files.\\
\section{Objectives}
\begin{itemize}
\item \textbf{Adaptability to Emerging Techniques:} Develop steganalysis models that can adapt to emerging steganographic methods by continuously learning and updating their knowledge base.
\item \textbf{Algorithmic Innovation:} Explore novel algorithms and methodologies tailored to analyze the intricate patterns and structures associated with image-in-image steganography.
\item \textbf{Enhanced Feature Extraction}: Investigate and optimize feature extraction techniques to capture subtle anomalies indicative of image-in-image concealment, ensuring high detection accuracy.
\item \textbf{Adaptability to Diverse Image Formats:}Develop steganalysis models capable of detecting concealed images across a variety of image formats, resolutions, and compression methods.
\item \textbf{Benchmarking and Evaluation:} Establish a comprehensive benchmark dataset containing images with various steganographic content for testing and evaluating the performance of developed steganalysis techniques.
\item \textbf{Development of Specialized Steganalysis, Techniques:} Create and apply steganalysis methods with a particular goal in mind: identifying images that are hidden inside of another.
\item \textbf{Algorithmic Innovation:} Investigate modern techniques and algorithms designed specifically to examine the complex structures and patterns connected to image-in-image steganography.
Flexibility to Diverse Image Formats: Create steganalysis models that can identify hidden images in a range of image formats, compression techniques, and resolutions.
\item \textbf{Real-time Implementation:} Explore the integration of the developed models into real-time steganalysis systems. Evaluate their performance in dynamic environments.
\item \textbf{Benchmarking and Evaluation:} To test and evaluate the effectiveness of created steganalysis tools with photos that have different steganographic content.
\end{itemize}

