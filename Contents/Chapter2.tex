\chapter{Literature Review}
Some work has been done in image steganalysis. Various steganalysis tools use different approaches like feature extraction, shallow ML, and deep learning methods to detect steganographically altered images. This literature review seeks to portray the history, methodologies, implementation and applications of steganalysis.\\
Multiple research has been done to achieve excellent results in steganalysis. Krzysztof Szczypiorski et al.\cite{1} used deep learning and ensemble classifiers to detect image steganography using different methods like DCTR and shallow machine learning classifiers. They found that performance depended heavily on the steganographic method used and on the density of the embedded hidden data. Detection of the content hidden with the nsF5 algorithm at the density 0.4 bpnzac was almost perfect while detection of data hidden using J-Uniward at 0.1 bpnzac was hardly possible. It is shown that steganalysis done using shallow ML is better in comparison to deep learning. This point is further proved by the fact that shallow ML consumes less resources and requires less time to be trained in comparison to deep ML and still provides accuracy similar or better than deep ML classifiers. \\
The document titled ``The Discrete Cosine Transform: Theory and Application''\cite{4} gives us a comprehensive overview of the Discrete Cosine Transform(DCT) and its application in digital image and video processing. The document discusses the properties of the DCT, including its decorrelation characteristics, energy compaction, and its ability to reduce entropy. It highlights the DCT's role in efficient coding and compression, particularly in the context of image and video standards such as JPEG and MPEG. Additionally, The document addresses the inverse DCT operation and its impact on visual distortion, providing examples of reconstructed images at different quantization levels.\\
\\George Berg et al.\cite{2} proposed an ML approach to steganalysis. This paper shows the feasibility of using a machine learning and data mining (ML/DM) approach to automatically build a steganography attack. This paper used three common data mining and learning techniques: decision trees, error back-propagation, artificial neural networks and the naïve Bayes classifier, to identify messages hidden in compression- (JPEG) and content based (GIF) images.\\
\\MT Hogan et al.\cite{3} evaluated the statistical limits by using probability density functions(pdfs). ML tests based on DC-DM are presented in this paper.To effectively uncover hidden information in images, we need a steganalysis tool with sharp pattern recognition skills. Sometimes, when we compare images that have been manipulated with certain tools to their original versions, we can spot a few noticeable visual irregularities – like odd pixels or changes in dimensions due to cropping or padding. If an image doesn't fit specific size criteria, it might get cropped or padded, and you'll see black spaces. Interestingly, most manipulated images don't give away obvious clues when compared to their originals. The simplest clue is a size increase between the manipulated and original images. Other signatures show up in how the colors are arranged in the image, such as a significant change in the number of colors or a gradual increase or decrease. Grayscale images follow a different pattern, increasing incrementally. Another strong indicator is an unusual number of black shades in a grayscale image.\\
`Steganalysis in high dimensions: Fusing classifiers built on random subspace`\cite{8} provides core concepts of this project such as ensemble classifier and importance of selection of features. A distinctive subject which it has touched upon is the concept of Curse of Dimensionality (CoD) which shows the relation of complexity and increase in  resource usage for computation. It is highlighted how ensemble classifiers can counter this problem by using reduced dimension for training its base learners.\\
`Ensemble Classifiers for Steganalysis of Digital Media`'\cite{5} highlights several key studies in the field of steganalysis, which provides a solid foundation for understanding the current state of steganalysis. The document discusses the implementation of ensemble based steganographically altered image classifier using many base learners for classification. The proposed base learners are trained using FLD analysis due to its ability to increase diversity The performance of the proposed model even though gets trained in very less time in comparison to usually used classification method of G-SVM can classify with similar or better accuracy. It is highlighted that a G-SVM classifier takes about 8 hours to be properly trained while an ensemble classifier takes only 20 minutes.\\                   
`A fast and accurate steganalysis using Ensemble classifiers`\cite{6} provides an in-depth insight into the use of an ensemble of classifiers for steganalysis, with a focus on machine learning. The ensemble-based steg analyzer uses feature vectors from multiple stegalyzers to create a decision algorithm that allows the combination of information from different steganalyzers. The resulting steganalyzer is also inherently suitable for multi-class classification scenarios. The paper presents a novel steganalysis decision framework using hierarchical classifiers, which addresses the limitations of existing steganalysis methods and provides a scalable and cost-effective approach to steganalysis. Ensemble classifiers are designed to overcome the limitations of individual classifiers by combining their outputs to achieve better performance. Steganalysis using ensemble classifiers is a powerful approach that utilizes the strength of multiple classifiers to help improve the detection of hidden information in images. It provides diverse steganographic techniques while also enhancing the overall accuracy. Ensemble classifiers are designed to overcome the limitations of individual classifiers by combining their outputs, thereby achieving better performance.\\
`J. Kodovský and J. Fridrich. Calibration revisited`\cite{9} provide information on the pre features and their Cartesian calibrated and Non-cartesian calibrated form.`A Markov Process Based Approach to Effective Attacking JPEG Steganography`\cite{10} and `Merging Markov and DCT features for multi-class JPEG steganalysis`\cite{11}  guides the outlook of our project to a better angle as it provides very crucial details on the section of feature extraction.They provide more insight on the pre features which can be utilized for better classification. These literature provided more insights on CC-PEv and CC-SHI which are different pre features used for steganalysis. “JPEG Image Steganalysis Utilizing both Intrablock and Interblock Correlations”  provides more insight on the importance of considering relation between inter and intra block correlations during pre feature creation for better detection or classification. \\
The dataset we will be using on this project will be taken from IStego100k\cite{7}. IStego 100K is a large-scale steganalysis consisting of 208,104 images with a size of 1024*1024 pixels. The training set consists of 200,000 images organized into 100,000 cover-setgo image pairs. The testing set comprises the remaining 8,104 images. Each image in the dataset has randomly assigned quality factors in the range of 75-95. Three well-known steganographic algorithms J-uniward, nsF5, and UERD\cite{}\cite{}\cite{} are randomly selected for embedding in the images. The embedding rate for each image is randomly set in the range of 0.1-0.4 bpac. 

The relevant papers that we studied to grab knowledge about this project are given in the review matrix below:

 \begin{table}[h!]
    \hbadness=99999 
    \begin{tabular}{|p{2.5cm}|p{2.5cm}|p{2.5cm}|p{2.5cm}|p{4cm}|}
    \hline
    S.N& Title& Authors& Year & Keywords\\
    \hline
    1&Dection of Image Steganography using deep learning and ensemble classifiers&Mikołaj Płachta, Marek Krzemie`n, Krzysztof Szczypiorski, and Artur Janicki.&2022& Ensemble Classifier,BOSS Database,steganalysis, Deep Learning\\
    \hline
    2&Searching For Hidden Messages: Automatic detection of steganography&George Berg, Ian Davidson, Ming-Yuan Duan and Goutam Paul&2003& Desicion Tree,error back-propagation artificial neural networks and the naïve Bayes classifier\\
    \hline
    3&ML detection of steganography&Mark T. Hogan, Neil J. Hurley, Gu`enol`e C.M. Silvestre, F`elix Balado and Kevin M. Whelan&2005&Security Automation\\
    \hline
    4&The Discrete Cosine Transform: Theory and Application&Kodovsky, Jan and Fridrich, Jessica and Holub, Vojtech&2003&DCT, Image processing, DFT\\
    \hline
    5&Ensemble Classifiers for Steganalysis of Digital Media&Syed Ali Khayam&2012&Feature Construction, DCT Coefficients, Support Vector Machine(SVMs,)\\
    \hline
    6&A fast and accurate steganalysis using Ensemble classifiers &Torkaman, Arezoo and Safabakhsh, Reza&2013&Ensemble Classifier, Fisher's Linear Discriminant(FLD)\\
    \hline
    \end{tabular}
    \caption{Review Matrix with Research Papers, authors and purpose}
\end{table}
