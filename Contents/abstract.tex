\begin{abstract}
\addcontentsline{toc}{chapter}{Abstract}
\thispagestyle{plain}
\pagenumbering{roman}
\setcounter{page}{1}
Steganography is a covert visual attack, that involves hiding malicious data inside innocent-looking carrier information. It's a technique for hiding information within an image, audio, or video file in such a way that the hidden information is not readily apparent to the human eye or ear. Digital images are the most common carrier format for steganography due to their frequent use on social media, websites, and email. Almost two-thirds of internet is made up of JPEGs, which serve as perfect carriers for these types of malware. Hence, it is important to have strong steganalysis methods. Our paper focuses on using ensemble classifiers as well as Generative Adversarial Networks (GAN) to detect hidden malicious contents in carrier files.  Ensemble classifiers is made up of various models working independently, employed to identify images modified using various steganography algorithms. These models are integrated into another model which utilizes algorithms such as logistic regression, to detect the presence or absence of malicious data. Generative Adversarial Networks (GAN) plays a crucial role in generating steganographically modified images, serving as a testing ground to ensure the effectiveness of the ensemble classifier models in successfully detecting and analyzing potential threats of steganogrpahically modified carriers. The core objective is to enhance steganalysis accuracy by integrating Machine Learning algorithms to counter the ever evolving field of steganography.
\end {abstract}

