\chapter*{Abstract} 
\addcontentsline{toc}{chapter}{Abstract}
\thispagestyle{plain}
\pagenumbering{roman}
\setcounter{page}{1}
%\begin{sloppypar}
\sloppy
Steganography is a technique used to hide information, such as text, images, audio or videos, within another carrier file in a way that is not easily detectable by human senses. Steganography is a tool that can be exploited by malicious software as a means to conceal malicious information and potentially attack systems. Malware can perform various harmful actions, such as damaging data, stealing information, or disrupting the normal operation of a computer system. Almost two-thirds of the internet is made up of JPEGs, this makes digital images the most common carrier format for steganography. Ensemble classifiers created using a bagging ensemble method implemented as a random forest are used to detect hidden malicious content from images as carrier files. Robustness, Model diversity and a faster development cycle were the main reasons for choosing the Ensemble classifier over other Machine learning approaches. The proposed model will perform effectively on three different steganographic methods [nsF5, J-UNIWARD and UERD]  that hide messages in JPEG images.\\ \\
\normalsize{\textbf{Keywords:} Steganography, Steganalysis, Ensemble Classifiers, \mbox{Machine Learning}, Random forest, JPEG }
%\end{sloppypar}


