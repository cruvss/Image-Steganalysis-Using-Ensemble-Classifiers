\chapter{Feasibility study} \sloppy
After the problem is clearly understood and solutions proposed, the next step is to conduct the
feasibility study. Feasibility study is defined as evaluation or analysis of the potential impact
of a proposed project or program. The objective is to determine whether the proposed system
is feasible. There are three aspects of feasibility study which are discussed below.\\ \\
\hbadness=99999 
\textbf{Technical Feasibility:}\\For the technical part, we're getting our project data from IStego100K dataset which contain 200,000 images. These images have been modified using three different algorithms which creates diversity in the dataset used improving the reliability of the system. We're using free software to build the project, and the department is providing cloud resources like RAM and GPU for training our model. This setup makes sure our project is doable and integrates well with the currently existing system. Thus, we can conclude that it is technically feasible.\\ \\
\textbf{Economical Feasibility:}\\The only cost for the project is the computational power, covering processing and electricity. Since the department will be providing the processing power needed to train the model, the cost is almost zero.\\ \\
\textbf{Operational Feasibility:}\\We have decided to use the Shallow ML approach which allows the model to be trained with less computational power in comparison to deep learning. For shallow machine learning we are planning to implement an ensemble classifier and each of its models will be trained using FLD to improve its effectiveness. Deep learning implements the CNN approach which requires higher computational power to be trained. Thus, we decided to use a simpler machine learning approach that doesn't need a lot of computational power, unlike the more complex deep learning method called Convolutional Neural Network (CNN). After we train the system, it's ready to use and can easily be added to a webpage or any other interface. This way, the system is practical and doesn't need a lot of resources making it able to be effectively implemented in real-life applications making it operationally feasible.
\\