\chapter*{Abstract} 
\addcontentsline{toc}{chapter}{Abstract}
\thispagestyle{plain}
\pagenumbering{roman}
\setcounter{page}{1}
%\begin{sloppypar}
\sloppy
Steganography is a technique used to hide information, such as text, images, audio or videos, within another carrier file in a way that is not easily detectable by human senses. Digital images, constituting nearly two-thirds of online content are popular carriers for steganography.This can be exploited by malicious actors to hide harmful information. This hidden malware can damage data, steal information, or disrupt systems. To address this challenge, we propose an ensemble classifier model using bagging method implemented as a random forest. We will be using high dimensional features from CC-C300 model to train our ensemble classifier. This approach offers robustness, model diversity, and faster development compared to other machine learning methods. The proposed model will perform effectively on three different steganographic methods; nsF5, J-UNIWARD and UERD, that hide messages in JPEG images.\\ \\
\normalsize{\textbf{Keywords:} Steganography, Steganalysis, Ensemble Classifiers, CC-C300, \mbox{Machine Learning}, Random forest, JPEG }
%\end{sloppypar}


